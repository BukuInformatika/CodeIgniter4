\section{CodeIgniter}
Codeigniter merupakan suatu Web Application Framework (WAF) yang di bentuk khusus untuk mempermudah para developer web baik ahli maupun bagi kaum awam dalam mengembangkan dan membuat apilkasi berbasis web.
\par
Codeigniter itu sendiri berisi beberapa kumpulan kode berupa pustaka (library) dan alat (tools) yang dipadukan/digabungkan sedemikian rupa menjadi suatu kerangka kerja (framework). Dalam Codeigniter yang merupakan framework web untuk Bahasa pempograman PHP terkhususnya merupakan buatan yang dibuat oleh Rick Ellis pada tahun 2006. Yang juga merupakan penemu dan pendiri dari EllisLab.
\par
EllisLab itu sendiri merupakan suatu tim kerja yang berdiri pada tahun 2002 dan bergerak di bidang pembuatan software dan tool untuk para pengembang web. Dan sejak tahun 2014 sampai sekarang, ElLisLab telah menyerahkan hak kepemilikan Codeigniter terhadap British Columbia Institute of Technology (BCIT) untuk proses pengembangan lebih lanjut kedepannya.
\par
Pada saat ini situs web resmi dari Codeigniter telah berubah dari www.ellislab.com menjadi www.codeigniter.com. Sehingga dapat mempermudah dalam pencarian mengenai fitur-fiturnya. Codeigniter memiliki banyak fitur  yang membantu dalam pengembangan PHP untuk dapat membuat aplikasi web secara mudah, cepat dan efisien.
\par
Dibandingkan dengan framework web PHP lainnya, Codeigniter memiliki desain yang lebih sederhana dan bersifat fleksibel (tidak kaku). Dan Codeigniter mengizinkan para pengembang untuk menggunakan framework secara parsial atau secara keseluruhan. Sehingga codeigniter masih memberi kebebasan kepada pengembang untuk menulis bagian-bagian kode tertentu di dalam aplikasi menggunakan cara konvensional (tanpa framework).
\par
Codeigniter menganut arsitektur Model-View-Controller (MVC), yang memisahkan antar bagian kode untuk penanganan proses bisnis dengan bagian kode untuk keperluan presentasi (tampilan). Dengan menggunakan pola desain ini, memungkin para pengembang untuk mengerjakan aplikasi berbasis web secara bersama (teamwork). para pengembang web lebih bisa berfokus pada bagian masing-masing tanpa mengganggu bagian yang lain. Sehingga aplikasi yang dibangun akan selesai lebih cepat dan menghemat waktu yang digunakan. 

\section{Fungsi CodeIgniter}
CodeIgniter juga memiliki dokumentasi yang super lengkap disertai dengan contoh implementasi kodenya. Dokumentasi yang lengkap inilah yang menjadi salah satu alasan kuat mengapa banyak orang memilih CodeIgniter sebagai framework pilihannya. Fungsi dari CodeIgniter diataranya adalah sebagai beriku :

\begin{enumerate}
\item Mempercepat dan mempermudah kita dalam pembuatan website.
\item Menghasilkan struktur pemrograman yang sangat rapi, baik dari segi kode maupun struktur file phpnya.
\item Memberikan standar coding sehingga memudahkan kita atau orang lain untuk mempelajari kembali system aplikasi yang dibangun.
\end{enumerate}  

\section{Kelebihan CodeIgniter}
CodeIgniter memiliki beberapa kelebihan dibandingkan dengan framework lainnya diantaranya adalah sebagai berikut :

\begin{enumerate}
\item Berukuran sangat kecil. File download nya hanya sekitar 2MB, itupun sudah includedokumentasinya yang sangat lengkap.
\item Dokumentasi yang bagus. Saat anda mendownloadnya, telah disertakan dengan dokumentasi yang berisi pengantar, tutorial, bagaimana panduan penggunaan, serta referensi dokumentasi untuk komponen-komponennya.
\item Kompitabilitas dengan Hosting. CodeIgniter mampu berjalan dengan baik pada hampirsemua platfom hosting. CodeIgniter juga mendukung database-database paling umum, termasuk MySQL.
\item Tidak ada aturan coding yang ketat. Terserah anda jika anda hanya ingin menggunakan Controller, tanpa View, atau tidak menggunakan Model, atau tidak salah satu keduanya. Namun dengan menggunakan ketiga komponennya adalah pilihan lebih bijak.
\item Kinerja yang baik. Codeigniter sangat cepat bahkan mungkin bisa dibilang merupakan framework yang paling cepat yang ada saat ini.
\item Sangat mudah diintegrasikan. CodeIgniter sangat mengerti tentang pengembangan berbagai library saat ini. Karenanya CodeIgniter memberikan kemudahan untuk diintegrasikan dengan library-library yang tersedia saat ini.
\item Sedikit Konfigurasi. Konfigurasi CodeIgniter terletak di folder aplication/config. CodeIgniter tidak membutuhkan konfigurasi yang rumit, bahkan untuk mencoba menjalankannya, tanpa melakukan konfigurasi sedikitpun ia sudah bisa berjalan.
\item Mudah dipelajari. Disamping dokementasi yang lengkap, ia juga memiliki berbagai forum diskusi.
\end{enumerate}  

\section{Kelebihan CodeIgniter}
Ada kelebihan dan juga pasti ada kekurangan pada CodeIgniter kekurangan yang dimiliki merupakan beberapa pendapat orang-orang yang pernah mengunakannya dan kekurangan ini yang menunjang untuk adanya perbaikan kedepannya, dan kekurangan dari CodeIgniter itu sendiri adalah sebagai berikut :

\begin{enumerate}
\item CodeIgniter tidak ditujukan untuk pembuatan web dengan skala besar.
\item Library yang sangat terbatas. Hal ini dikarenakan sangat sulit mencari plugin tambahan yang terverifikasi secara resmi, karena pada situsnya CodeIgniter tidak menyediakan plugin-plugin tambahan untuk mendukung pengembangan aplikasi dengan CI.
\item Belum adanya editor khusus CodeIgniter, sehingga dalam melakukan create project dan modul-modulnya harus berpindah-pindah folder.
\end{enumerate}  