\section{CodeIgniter}
Codeigniter merupakan suatu Web Application Framework (WAF) yang di bentuk khusus untuk mempermudah para developer web baik ahli maupun bagi kaum awam dalam mengembangkan dan membuat apilkasi berbasis web.
\par
Codeigniter itu sendiri berisi beberapa kumpulan kode berupa pustaka (library) dan alat (tools) yang dipadukan/digabungkan sedemikian rupa menjadi suatu kerangka kerja (framework). Dalam Codeigniter yang merupakan framework web untuk Bahasa pempograman PHP terkhususnya merupakan buatan yang dibuat oleh Rick Ellis pada tahun 2006. Yang juga merupakan penemu dan pendiri dari EllisLab.
\par
EllisLab itu sendiri merupakan suatu tim kerja yang berdiri pada tahun 2002 dan bergerak di bidang pembuatan software dan tool untuk para pengembang web. Dan sejak tahun 2014 sampai sekarang, ElLisLab telah menyerahkan hak kepemilikan Codeigniter terhadap British Columbia Institute of Technology (BCIT) untuk proses pengembangan lebih lanjut kedepannya.
\par
Pada saat ini situs web resmi dari Codeigniter telah berubah dari www.ellislab.com menjadi www.codeigniter.com. Sehingga dapat mempermudah dalam pencarian mengenai fitur-fiturnya. Codeigniter memiliki banyak fitur  yang membantu dalam pengembangan PHP untuk dapat membuat aplikasi web secara mudah, cepat dan efisien.
\par
Dibandingkan dengan framework web PHP lainnya, Codeigniter memiliki desain yang lebih sederhana dan bersifat fleksibel (tidak kaku). Dan Codeigniter mengizinkan para pengembang untuk menggunakan framework secara parsial atau secara keseluruhan. Sehingga codeigniter masih memberi kebebasan kepada pengembang untuk menulis bagian-bagian kode tertentu di dalam aplikasi menggunakan cara konvensional (tanpa framework).
\par
Codeigniter menganut arsitektur Model-View-Controller (MVC), yang memisahkan antar bagian kode untuk penanganan proses bisnis dengan bagian kode untuk keperluan presentasi (tampilan). Dengan menggunakan pola desain ini, memungkin para pengembang untuk mengerjakan aplikasi berbasis web secara bersama (teamwork). para pengembang web lebih bisa berfokus pada bagian masing-masing tanpa mengganggu bagian yang lain. Sehingga aplikasi yang dibangun akan selesai lebih cepat dan menghemat waktu yang digunakan.   


