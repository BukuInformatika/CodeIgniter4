%%%%%%%%%%%%%%
%% Run LaTeX on this file several times to get Table of Contents,
%% cross-references, and citations.

%% If you have font problems, you may edit the w-bookps.sty file
%% to customize the font names to match those on your system.

%% w-bksamp.tex. Current Version: Feb 16, 2012
%%%%%%%%%%%%%%%%%%%%%%%%%%%%%%%%%%%%%%%%%%%%%%%%%%%%%%%%%%%%%%%%
%
%  Sample file for
%  Wiley Book Style, Design No.: SD 001B, 7x10
%  Wiley Book Style, Design No.: SD 004B, 6x9
%
%
%  Prepared by Amy Hendrickson, TeXnology Inc.
%  http://www.texnology.com
%%%%%%%%%%%%%%%%%%%%%%%%%%%%%%%%%%%%%%%%%%%%%%%%%%%%%%%%%%%%%%%%

%%%%%%%%%%%%%
% 7x10
%\documentclass{wileySev}

% 6x9
\documentclass{wileySix}

\usepackage{graphicx}
\usepackage{listings}

\usepackage{color}
 
\definecolor{codegreen}{rgb}{0,0.6,0}
\definecolor{codegray}{rgb}{0.5,0.5,0.5}
\definecolor{codepurple}{rgb}{0.58,0,0.82}
\definecolor{backcolour}{rgb}{0.95,0.95,0.92}
 
\lstdefinestyle{mystyle}{
    backgroundcolor=\color{backcolour},   
    commentstyle=\color{codegreen},
    keywordstyle=\color{magenta},
    numberstyle=\tiny\color{codegray},
    stringstyle=\color{codepurple},
    basicstyle=\footnotesize,
    breakatwhitespace=false,         
    breaklines=true,                 
    captionpos=b,                    
    keepspaces=true,                 
    numbers=left,                    
    numbersep=5pt,                  
    showspaces=false,                
    showstringspaces=false,
    showtabs=false,                  
    tabsize=2,
    language=sh
}
 
\lstset{style=mystyle}

%%%%%%%
%% for times math: However, this package disables bold math (!)
%% \mathbf{x} will still work, but you will not have bold math
%% in section heads or chapter titles. If you don't use math
%% in those environments, mathptmx might be a good choice.

% \usepackage{mathptmx}

% For PostScript text
\usepackage{w-bookps}

%%%%%%%%%%%%%%%%%%%%%%%%%%%%%%%%%%%%%%%%%%%%%%%%%%%%%%%%%%%%%%%%
%% Other packages you might want to use:

% for chapter bibliography made with BibTeX
% \usepackage{chapterbib}

% for multiple indices
% \usepackage{multind}

% for answers to problems
% \usepackage{answers}

%%%%%%%%%%%%%%%%%%%%%%%%%%%%%%
%% Change options here if you want:
%%
%% How many levels of section head would you like numbered?
%% 0= no section numbers, 1= section, 2= subsection, 3= subsubsection
%%==>>
\setcounter{secnumdepth}{3}

%% How many levels of section head would you like to appear in the
%% Table of Contents?
%% 0= chapter titles, 1= section titles, 2= subsection titles, 
%% 3= subsubsection titles.
%%==>>
\setcounter{tocdepth}{2}

%% Cropmarks? good for final page makeup
%% \docropmarks

%%%%%%%%%%%%%%%%%%%%%%%%%%%%%%
%
% DRAFT
%
% Uncomment to get double spacing between lines, current date and time
% printed at bottom of page.
% \draft
% (If you want to keep tables from becoming double spaced also uncomment
% this):
% \renewcommand{\arraystretch}{0.6}
%%%%%%%%%%%%%%%%%%%%%%%%%%%%%%

%%%%%%% Demo of section head containing sample macro:
%% To get a macro to expand correctly in a section head, with upper and
%% lower case math, put the definition and set the box 
%% before \begin{document}, so that when it appears in the 
%% table of contents it will also work:

\newcommand{\VT}[1]{\ensuremath{{V_{T#1}}}}

%% use a box to expand the macro before we put it into the section head:

\newbox\sectsavebox
\setbox\sectsavebox=\hbox{\boldmath\VT{xyz}}

%%%%%%%%%%%%%%%%% End Demo


\begin{document}


\booktitle{Cerdas Menguasai Gittt}
\subtitle{Dalam 24 Jam}

\authors{Rolly M. Awangga\\
\affil{Informatics Research Center}
%Floyd J. Fowler, Jr.\\
%\affil{University of New Mexico}
}

\offprintinfo{Cerdas Menguasai Git, First Edition}{Rolly M. Awangga}

%% Can use \\ if title, and edition are too wide, ie,
%% \offprintinfo{Survey Methodology,\\ Second Edition}{Robert M. Groves}

%%%%%%%%%%%%%%%%%%%%%%%%%%%%%%
%% 
\halftitlepage

\titlepage


\begin{copyrightpage}{2019}
%Survey Methodology / Robert M. Groves . . . [et al.].
%\       p. cm.---(Wiley series in survey methodology)
%\    ``Wiley-Interscience."
%\    Includes bibliographical references and index.
%\    ISBN 0-471-48348-6 (pbk.)
%\    1. Surveys---Methodology.  2. Social 
%\  sciences---Research---Statistical methods.  I. Groves, Robert M.  II. %
%Series.\\
%
%HA31.2.S873 2007
%001.4'33---dc22                                             2004044064
\end{copyrightpage}

\dedication{`Jika Kamu tidak dapat menahan lelahnya belajar, 
Maka kamu harus sanggup menahan perihnya Kebodohan.'
~Imam Syafi'i~}

\begin{contributors}
\name{Rolly Maulana Awangga,} Informatics Research Center., Politeknik Pos Indonesia, Bandung,
Indonesia



\end{contributors}

\contentsinbrief
\tableofcontents
\listoffigures
\listoftables
\lstlistoflistings


\begin{foreword}
Sepatah kata dari Kaprodi, Kabag Kemahasiswaan dan Mahasiswa
\end{foreword}

\begin{preface}
Buku ini diciptakan bagi yang awam dengan git sekalipun.

\prefaceauthor{R. M. Awangga}
\where{Bandung, Jawa Barat\\
Februari, 2019}
\end{preface}


\begin{acknowledgments}
Terima kasih atas semua masukan dari para mahasiswa agar bisa membuat buku ini 
lebih baik dan lebih mudah dimengerti.

Terima kasih ini juga ditujukan khusus untuk team IRC yang 
telah fokus untuk belajar dan memahami bagaimana buku ini mendampingi proses 
Intership.
\authorinitials{R. M. A.}
\end{acknowledgments}

\begin{acronyms}
\acro{ACGIH}{American Conference of Governmental Industrial Hygienists}
\acro{AEC}{Atomic Energy Commission}
\acro{OSHA}{Occupational Health and Safety Commission}
\acro{SAMA}{Scientific Apparatus Makers Association}
\end{acronyms}

\begin{glossary}
\term{git}Merupakan manajemen sumber kode yang dibuat oleh linus torvald.

\term{bash}Merupakan bahasa sistem operasi berbasiskan *NIX.

\term{linux}Sistem operasi berbasis sumber kode terbuka yang dibuat oleh Linus Torvald
\end{glossary}

\begin{symbols}
\term{A}Amplitude

\term{\hbox{\&}}Propositional logic symbol 

\term{a}Filter Coefficient

\bigskip

\term{\mathcal{B}}Number of Beats
\end{symbols}

\begin{introduction}

%% optional, but if you want to list author:

\introauthor{Rolly Maulana Awangga, S.T., M.T.}
{Informatics Research Center\\
Bandung, Jawa Barat, Indonesia}

Pada era disruptif  \index{disruptif}\index{disruptif!modern} 
saat ini. git merupakan sebuah kebutuhan dalam sebuah organisasi pengembangan perangkat lunak.
Buku ini diharapkan bisa menjadi penghantar para programmer, analis, IT Operation dan Project Manajer.
Dalam melakukan implementasi git pada diri dan organisasinya.

Rumusnya cuman sebagai contoh aja biar keren\cite{awangga2018sampeu}.

\begin{equation}
ABC {\cal DEF} \alpha\beta\Gamma\Delta\sum^{abc}_{def}
\end{equation}

\end{introduction}

%%%%%%%%%%%%%%%%%%Isi Buku_

\chapter{Judul Bagian Pertama}
\section{CodeIgniter}
Codeigniter merupakan suatu Web Application Framework (WAF) yang di bentuk khusus untuk mempermudah para developer web baik ahli maupun bagi kaum awam dalam mengembangkan dan membuat apilkasi berbasis web.
\par
Codeigniter itu sendiri berisi beberapa kumpulan kode berupa pustaka (library) dan alat (tools) yang dipadukan/digabungkan sedemikian rupa menjadi suatu kerangka kerja (framework). Dalam Codeigniter yang merupakan framework web untuk Bahasa pempograman PHP terkhususnya merupakan buatan yang dibuat oleh Rick Ellis pada tahun 2006. Yang juga merupakan penemu dan pendiri dari EllisLab.
\par
EllisLab itu sendiri merupakan suatu tim kerja yang berdiri pada tahun 2002 dan bergerak di bidang pembuatan software dan tool untuk para pengembang web. Dan sejak tahun 2014 sampai sekarang, ElLisLab telah menyerahkan hak kepemilikan Codeigniter terhadap British Columbia Institute of Technology (BCIT) untuk proses pengembangan lebih lanjut kedepannya.
\par
Pada saat ini situs web resmi dari Codeigniter telah berubah dari www.ellislab.com menjadi www.codeigniter.com. Sehingga dapat mempermudah dalam pencarian mengenai fitur-fiturnya. Codeigniter memiliki banyak fitur  yang membantu dalam pengembangan PHP untuk dapat membuat aplikasi web secara mudah, cepat dan efisien.
\par
Dibandingkan dengan framework web PHP lainnya, Codeigniter memiliki desain yang lebih sederhana dan bersifat fleksibel (tidak kaku). Dan Codeigniter mengizinkan para pengembang untuk menggunakan framework secara parsial atau secara keseluruhan. Sehingga codeigniter masih memberi kebebasan kepada pengembang untuk menulis bagian-bagian kode tertentu di dalam aplikasi menggunakan cara konvensional (tanpa framework).
\par
Codeigniter menganut arsitektur Model-View-Controller (MVC), yang memisahkan antar bagian kode untuk penanganan proses bisnis dengan bagian kode untuk keperluan presentasi (tampilan). Dengan menggunakan pola desain ini, memungkin para pengembang untuk mengerjakan aplikasi berbasis web secara bersama (teamwork). para pengembang web lebih bisa berfokus pada bagian masing-masing tanpa mengganggu bagian yang lain. Sehingga aplikasi yang dibangun akan selesai lebih cepat dan menghemat waktu yang digunakan.   




\chapter{Judul Bagian Kedua}
\section{CodeIgniter 4 pre alpha bukan CodeIgniter 3}
Semenjak BCIT mengambil CodeIgniter di bawah perlindungannya, mereka mencoba bergerak maju untuk memperbaiki beberapa permasalahan yang ada pada sebelumnya. CodeIgniter versi 3 dapat dikatakan sangat kompatibel dengan versi 2. Bahkan dengan mengubah modal di sana-sini Kita dapat memigrasi aplikasi dalam "jiffy". Pada saat ini CodeIgniter 4 telah dibuka untuk umum dalam versi pra-alfa akan tetapi mengenai source code tidak begitu lengkap dan bahkan perkembangannya cukup lama. 
\par
Hal pertama yang saya perhatikan adalah ada beberapa perubahan pada CodeIgniter sebelumnya. ini tidak terlihat seperti CodeIgniter seperti yang kita gunakan juga. BCIT memutuskan untuk membersihkan beberapa fungsi yang tidak begitu penting untuk membangun kembali sebuah framework yang lebih simple dan cepat di mengerti pada kalangan programmer. untuk pendokumentasinya dan juga perawatannya dilakukan perkembangan itu sendiri, butuh beberapa saat untuk mencari tahu bagaimana semuanya cocok pada Framework yang disebut CodeIgniter 4, yang hanya akan berjalan mulai dengan PHP versi 7. Jadi pertama-tama cari tahu apakah penyedia host Kita mendukung atau jika sudah diinstal dan berjalan di server lokal Kita sendiri. CodeIgniter 4 TIDAK kompatibel dengan CodeIgniter 3.
\par
Semuanya sekarang berhubungan dengan ruang nama. Tidak ada lagi $ this-> load-> model ('sesuatu')$  tetapi Kita harus terbiasa dengan "penggunaan" dan kembali memangkas referensi ruang nama.

\section{apa yang masih ada? dan apa yang telah berubah?}
Dalam ulasan kecil dan tidak lengkap ini, saya tidak akan berbicara tentang "pandangan", karena saya belum sampai sejauh itu dan saya menggunakan CodeIgniter terutama sebagai server REST ke aplikasi berbasis Ext JS. Saya akan berbicara tentang "pandangan" setelah benar-bener CodeIgniter 4 di rilis dan bias di publikasikan sepenuhnya.
CodeIgniter masih merupakan kerangka kerja MVC (model-view-controller) dan tim pengembangan sangat jelas tentang HMVC (hierarki-model-tampilan-controller), itu tidak akan diimplementasikan. HMVC itu sendiri merupakan kumpulan triad MVC tradisional yang beroperasi sebagai satu aplikasi. Setiap triad sepenuhnya independen dan dapat dieksekusi tanpa kehadiran yang lain.
\par
Apa yang akan Kita perhatikan adalah struktur jalur yang berbeda. Kita sekarang akan menemukan jalur "aplikasi", "sistem" dan "publik". Itu perlu penjelasan. Gagasan di balik semua ini adalah bahwa jalur "publik" harus menjadi jalur root dari aplikasi web Kita. Ini berarti bahwa "sistem" dan "aplikasi" akan disembunyikan dari mata publik. Ini meningkatkan keamanan. Juga di belakang mata publik adalah folder "dapat ditulis", yang memiliki akses tulis, tetapi tidak dapat diakses langsung oleh URL.

\subsection{Tidak ada perbedaan dalam memuat objek}
CodeIgniter melakukan pemuatan Model, Perpustakaan, dan objek lainnya dengan cara yang sama. Memuat Model atau Pustaka secara otomatis, serta objek lain, dapat diatur di dalam "Autoload.php" di folder "config". Di sana Kita memiliki array $ psr4 dan array $ classmap , yang dapat digunakan untuk memuat suatu objek secara otomatis. Tidak ada perbedaan antara kelas ketika datang ke pemuatan otomatis. Kita harus memberi tahu apa namespace itu dan di mana ia dapat ditemukan.
$classmap = [
   'TMDB_API' => APPPATH . 'ThirdParty/tmdb-v3.php'
];$
Sampel tersebut menunjukkan bagaimana namespace dibuat untuk objek non-CodeIgniter di folder "ThirdParty" di dalam folder "application".

\subsection{Memuat Model dan Perpustakaan dalam pengontrol}
Kutipan di bawah ini  menunjukkan bagaimana Kita memuat model dan pustaka di controller.
Sampel memiliki kelas model bernama "TmdbModel" dan kelas perpustakaan bernama "Tmdb". Perhatikan deklarasi "gunakan". Kita harus terbiasa dengan itu (saya lakukan), tetapi Kita akan terbiasa dengan cepat. Saya harus mendeklarasikan antarmuka "CodeIgniter \ HTTP \" karena saya mendapatkan kesalahan pada konstruk induk ketika saya tidak menambahkan variabel $ request dan $ response . Meninggalkan parameter tidak berfungsi.





\bibliographystyle{IEEEtran} 
%\def\bibfont{\normalsize}
\bibliography{references}


%%%%%%%%%%%%%%%
%%  The default LaTeX Index
%%  Don't need to add any commands before \begin{document}
\printindex

%%%% Making an index
%% 
%% 1. Make index entries, don't leave any spaces so that they
%% will be sorted correctly.
%% 
%% \index{term}
%% \index{term!subterm}
%% \index{term!subterm!subsubterm}
%% 
%% 2. Run LaTeX several times to produce <filename>.idx
%% 
%% 3. On command line, type  makeindx <filename> which
%% will produce <filename>.ind 
%% 
%% 4. Type \printindex to make the index appear in your book.
%% 
%% 5. If you would like to edit <filename>.ind 
%% you may do so. See docs.pdf for more information.
%% 
%%%%%%%%%%%%%%%%%%%%%%%%%%%%%%

%%%%%%%%%%%%%% Making Multiple Indices %%%%%%%%%%%%%%%%
%% 1. 
%% \usepackage{multind}
%% \makeindex{book}
%% \makeindex{authors}
%% \begin{document}
%% 
%% 2.
%% % add index terms to your book, ie,
%% \index{book}{A term to go to the topic index}
%% \index{authors}{Put this author in the author index}
%% 
%% \index{book}{Cows}
%% \index{book}{Cows!Jersey}
%% \index{book}{Cows!Jersey!Brown}
%% 
%% \index{author}{Douglas Adams}
%% \index{author}{Boethius}
%% \index{author}{Mark Twain}
%% 
%% 3. On command line type 
%% makeindex topic 
%% makeindex authors
%% 
%% 4.
%% this is a Wiley command to make the indices print:
%% \multiprintindex{book}{Topic index}
%% \multiprintindex{authors}{Author index}

\end{document}

